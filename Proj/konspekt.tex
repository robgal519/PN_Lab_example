\documentclass[12pt]{article}
% \usepackage[polish]{babel}
\usepackage{polski} \usepackage{natbib} \usepackage{url}
\usepackage[utf8]{inputenc} \usepackage{amsmath} \usepackage{graphicx}
\usepackage{parskip} \usepackage{fancyhdr} \usepackage{vmargin} \setmarginsrb{3
  cm}{2.5 cm}{3 cm}{2.5 cm}{1 cm}{1.5 cm}{1 cm}{1.5 cm} \usepackage{listings}

\lstset{basicstyle=\ttfamily, showstringspaces=false, commentstyle=\color{red},
  keywordstyle=\color{blue} }

\title{Binutils, biblioteki statyczne i dynamiczne} % Title
\author{} % Author
\date{15 V 2018} % Date

\makeatletter \let\thetitle\@title \let\theauthor\@author \let\thedate\@date
\makeatother

\pagestyle{fancy} \fancyhf{} \rhead{\theauthor} \lhead{\thetitle}
\cfoot{\thepage}

\begin{document}

%%%%%%%%%%%%%%%%%%%%%%%%%%%%%%%%%%%%%%%%%%%%%%%%%%%%%%%%%%%%%%%%%%%%%%%%%%%%%%%%%%%%%%%%%

\begin{titlepage}
  \centering \vspace*{0.5 cm} \includegraphics[scale = 0.75]{agh.jpg}\\[1.0
  cm] % University Logo
  % \textsc{\LARGE Akademia Górniczo-Hutnicza im. Stanisława Staszica w
  % Krakowie}\\[2.0 cm] % University Name
  \textsc{\Large Programowanie Niskopoziomowe}\\[0.5 cm] % Course Code
  \textsc{\large Konspekt Laboratoryjny}\\[0.5 cm] % Course Name
  \rule{\linewidth}{0.2 mm} \\[0.4 cm]
  { \huge \bfseries \thetitle}\\
  \rule{\linewidth}{0.2 mm} \\[1.5 cm]

  \begin{minipage}{0.4\textwidth}
    \begin{flushleft} \large \vspace{3cm}
      \emph{Autorzy:}\\
      Gabriel Górski\newline Robert Gałat % Your Student Number
    \end{flushleft}
  \end{minipage}~
  \begin{minipage}{0.4\textwidth}
    \begin{flushright} \large
    \end{flushright}
  \end{minipage}\\[2 cm]

  {\large \thedate}\\[2 cm]

  \vfill

\end{titlepage}

%%%%%%%%%%%%%%%%%%%%%%%%%%%%%%%%%%%%%%%%%%%%%%%%%%%%%%%%%%%%%%%%%%%%%%%%%%%%%%%%%%%%%%%%%

\tableofcontents
\pagebreak

%%%%%%%%%%%%%%%%%%%%%%%%%%%%%%%%%%%%%%%%%%%%%%%%%%%%%%%%%%%%%%%%%%%%%%%%%%%%%%%%%%%%%%%%%


\section{Informacje do zadań}
W tym miejscu zakładamy, że ogólnie rozumiana teoria z seminarium jest znana
uczestnikom laboratorium. Poniżej znajdują się rzeczy przydatne w wykonywaniu
zadań z laboratorium.
\subsection{Biblioteki statyczne}
Jeśli chcemy utworzyć bibliotekę statyczną, potrzebujemy pliki obiektowe które
mają się w niej znaleźć. Komenda wygląda następująco

\begin{lstlisting}
  ar rcs libNAZWA.a plik1.o plik2.o plik3.o
\end{lstlisting}
Żeby wykorzystać tą bibliotekę należy ją \textbf{zlinkować} z resztą plików
obiektowych (na tym etapie symbol \textbf{main} \textit{musi} się zawierać w
którymś z nich).

Wykorzystanie biblioteki: Flaga \textbf{-L} umożliwia dodanie ścieżek
przeszukiwań bibliotek dla linkera statycznego. Dla nagłówków istnieje
analogiczna flaga: \textbf{-I} --- przydatna na etapie kompilacji.

\begin{lstlisting}
  gcc -L . plik.o -o plik -lNAZWA
\end{lstlisting}

\subsection{Biblioteki współdzielone}
Podobnie ma się sprawa z bibliotekami dynamicznymi, tutaj jednak należy pamiętać
o większej ilości niuansów.

Po pierwsze składowe pliki obiektowe biblioteki należy skompilować z flagą
\textbf{fPIC}.

Tworzenie biblioteki:
\begin{lstlisting}
  gcc -shared plik1.o plik2.o plik3.o -o libNAZWA.so
\end{lstlisting}

Aby wykorzystać bibliotekę w czasie linkowania należy zrobić dokładnie to samo

\begin{lstlisting}
  gcc -L . plik.o -o plik -lNAZWA

\end{lstlisting}
\subsection{Zmienne środowiskowe dla LD}
Uruchomienie programu zbudowanego z wykorzystaniem bibliotek współdzielonych
wymaga ich obecności w czasie działania. Jeśli owa biblioteka znajduje się poza
domyślnymi katalogami przeszukiwań, wtedy musimy wykorzystać zmienne
środowiskowe które wpłyną na działanie LD. Aby dodać ścieżki przeszukiwań możemy
zrobić:

\begin{lstlisting}
  LD_LIBRARY_PATH=. ./plik
\end{lstlisting}

Dodatkowo, mając plik wykonywalny korzystający z bibliotek współdzielonych,
możemy zmusić linker dynamiczny by --- zanim zrobi wszystko wg reguł ustalonych
przy linkowaniu statycznym --- \textbf{z pierwszeństwem} załadował inne symbole.

Przykład:\\
\textit{Plik wykonywalny 'plik' korzysta z biblioteki współdzielonej
  dostarczającej symbol 'abc'; biblioteka SAMPLE dostarcza symbol 'abc' o takiej
  samej sygnaturze, ale o innej implementacji}

\begin{lstlisting}
  LD_PRELOAD=./libSAMPLE.so ./plik
\end{lstlisting}

W takiej sytuacji, wykorzystany zostanie symbol \textit{abc} z biblioteki SAMPLE
bo owa biblioteka została \textit{preload-owana} przed pozostałymi.

\subsection{Binutils}
Pliki binarne, a w szczególności obiektowe możemy analizować różnymi
narzędziami. Poniżej wymienione są wybrane tego typu programy

\begin{itemize}
\item \textbf{nm} --- listowanie symboli z pliku obiektowego
\item \textbf{objdump} --- w zależności od wybranych flag wyświetla w czytelny
  sposób określone informacje o fragmentach pliku obiektowego
\item \textbf{readelf} --- kompleksowe narzędzie do analizy struktury i
  zawartości pliku obiektowego
\end{itemize}

\subsection{Dynamiczne ładowanie}
Istnieje potężny i jeszcze bardziej elastyczny mechanizm umożliwiający ładowanie
bibliotek współdzielonych. Zamiast dodawać do pliku wykonywalnego informację o
bibliotekach na etapie linkowania statycznego lub wykorzystywać zmienne
środowiskowe, sam program może posiadać odpowiednie funkcjonalności do
manipulacji bibliotekami wedle własnych upodobań.

W systemach opierających się o jądro Linux takie możliwości dostarcza biblioteka
\textbf{dl} i komplementarny do niego nagłówek \textbf{dlfcn.h}.

Do wykonania zadań będziemy korzystać z następujących funkcji:
\begin{itemize}
\item
\begin{lstlisting}
void *dlopen (const char *__file, 
              int __mode)
\end{lstlisting}
  Umożliwa otworzenie biblioteki współdzielonej.
  \begin{itemize}
  \item \textbf{\_\_file} określa nazwę biblioteki współdzielonej którą chcemy
    wczytać
  \item \textbf{\_\_mode} określa zasadę doczytywania symboli (natychmiastową
    lub \textit{leniwą} tj. w razie potrzeby). Tutaj jako argument będziemy
    przekazywać \textbf{RTLD\_NOW}.
  \end{itemize}
  Wartość zwracana to jest \textit{handle} tj. referencja do biblioteki by móc
  się do niej odwoływać
\item
\begin{lstlisting}
int dlclose (void *__handle)
\end{lstlisting}
  Umożliwa zamknięcie biblioteki współdzielonej. Jako argument przyjmuje
  \textit{handle} do otwartej biblioteki którą chcemy zamknąć. Zwraca kod błędu.
\item
\begin{lstlisting}
void *dlsym (void *__restrict __handle, 
             const char *__restrict __name)
\end{lstlisting}
  Umożliwa pobranie symbolu z biblioteki
  \begin{itemize}
  \item \textbf{\_\_handle} określa \textit{handle} do biblioteki z której
    będziemy pobierać symbol. Dostępne są także \textit{pseudohandle's} jako
    makra określające specjalne zachowanie przy zawołaniu \textit{dlsym}. Np
    \textbf{RTLD\_NEXT} spowoduje pobranie wybranego symbolu z
    \textbf{następnej} biblioteki w kolejności wczytywania przez linker
    dynamiczny.
  \item \textbf{\_\_name} to nazwa pożądanego symbolu.
  \end{itemize}
  Wartość zwracana jest wskaźnikiem nieokreślonym na symbol. Jeśli ma
  wartość \textbf{NULL}, oznacza to błąd (brak symbolu etc).\\

  Jeśli symbolem jest funkcja, taki wskaźnik należy \textbf{zrzutować} na typ
  wskaźnikowy na funkcję o \textbf{odpowiedniej sygnaturze}

\item
\begin{lstlisting}
char *dlerror (void)
\end{lstlisting}
  Funkcja zwracająca stałą tekstową z informacjami o błędzie. Jeśli wszystko
  działa poprawnie zwraca \textbf{NULL}. Po zawołaniu następuje reset, co
  oznacza że kolejne zawołanie zwróci już NULL.
\end{itemize}

\section{Zadania}
\subsection{Biblioteki statyczne}
\begin{itemize}
\item Celem zadania jest uzupełnienie pliku \textbf{run.sh} tak, aby umożliwał
  on kompilację biblioteki statycznej, oraz zlinkowanie projektu do programu
  wynikowego. [\textbf{1a}]
\item Celem zadania jest uzupełnienie pliku \textbf{run.sh} tak, aby stworzyć
  biblioteki statyczne oraz zlinkować je z \textit{głównym} plikiem obiektowym.
  Czy zauważasz coś ciekawego? Jeśli tak, to czy potrafisz to wyjaśnić?
  [\textbf{1b}]
\end{itemize}
\subsection{Biblioteki współdzielone}
\begin{itemize}
\item W tym zadaniu należy utworzyć bibliotekę dynamiczną, zlinkować wobec niej
  plik obiektowy, a następnie otrzymany plik wykonywalny należy uruchomić ---
  pamiętaj o odpowiednich flagach kompilacji i linkowania! [\textbf{2}]
\item Celem zadania jest podmienienie implementacji funkcji która była w
  bibliotece z poprzedniego zadania.

  Należy to zrobić bez modyfikacji pliku wykonywalnego z poprzedniego
  zadania tj. poprzez wykorzystanie funkcjonalności linkera dynamicznego.

  Wprowadź własną implementację tej funkcji. [\textbf{3}]
\end{itemize}
\subsection{Binutils}
\begin{itemize}
\item W tym zadaniu należy dokonać kompilacji pliku \textit{relocatableFile.c} a
  następnie przeanalizować wygenerowany plik binarny programem \textit{nm} oraz
  \textit{objdump} [\textbf{4}]
\end{itemize}

\subsection{Pluginy i dynamiczne ładowanie}
\begin{itemize}
\item Celem zadania jest uzupełnienie pliku \textit{main.c} w taki sposób aby
  uruchomić funkcję z biblioteki libfoo.so, która powinna zostać załadowana w
  czasie działania programu. [\textbf{5}]

\item Celem zadania jest uzupełnienie brakujących części obsługi pluginów, oraz
  napisanie własnego pluginu, wzorując się na przygotowanym przykładzie

  Do uzupełnienia są następujące funkcje:[\textbf{6}]

  \begin{itemize}
  \item apply\_hook() \{PluginManager/PluginManager.c\}
  \item initPlugin() \{PluginManager/PluginLoader.c\}
  \end{itemize}
\end{itemize}

\end{document}
