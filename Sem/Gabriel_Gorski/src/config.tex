\documentclass[polish]{beamer} %If you want to create Polish presentation, replace 'english' with 'polish' and uncomment 3-th line, i.e., '\usepackage{polski}'
\usepackage[utf8]{inputenc}
\usepackage{polski} %Uncomment for Polish language
\usepackage{babel}
\usepackage{listings} %We want to put listings

\mode<beamer>{ 	%in 'beamer' mode
  % \hypersetup{pdfpagemode=FullScreen}		%Enable Full screen mode
  % \usetheme[parttitle=rightfooter]{AGH}		%Show part title in right footer
  \usetheme[nosidebar]{AGH}			%Do not show sidebar on non-title slides
  % \usetheme[nosidebar,margins=1em]{AGH}		%Do not show sidebar on non-title slidesa and set both margins (left / right) to 1em
  % \usetheme[dark]{AGH}                 		%Use dark background
  % \usetheme[dark,parttitle=leftfooter]{AGH}  	%Use dark background and show part title in left footer
}
\mode<handout>{	%in 'handout' mode
  \hypersetup{pdfpagemode=None}		
  \usepackage{pgfpages}
  \pgfpagesuselayout{4 on 1}[a4paper,border shrink=5mm,landscape]	%show 4 slides on 1 page
  \usetheme{boxes}
  \addheadbox{structure}{\quad\insertpart\hfill\insertsection\hfill\insertsubsection\qquad} 	%content of header
  \addfootbox{structure}{\quad\insertauthor\hfill\insertframenumber\hfill\insertsubtitle\qquad} 	%content of footer
}

\AtBeginPart{ %At begin part: display its name
  \frame{\partpage}
} 


%%%%%%%%%%% Configuration of the listings package %%%%%%%%%%%%%%%%%%%%%%%%%%
% Source: https://en.wikibooks.org/wiki/LaTeX/Source_Code_Listings#Using_the_listings_package
%%%%%%%%%%%%%%%%%%%%%%%%%%%%%%%%%%%%%%%%%%%%%%%%%%%%%%%%%%%%%%%%%%%%%%%%%%%% 
\lstset{ %
  backgroundcolor=\color{white},   % choose the background color
  basicstyle=\footnotesize,        % the size of the fonts that are used for the code
  breakatwhitespace=false,         % sets if automatic breaks should only happen at whitespace
  breaklines=true,                 % sets automatic line breaking
  captionpos=b,                    % sets the caption-position to bottom
  commentstyle=\color{green},      % comment style
  deletekeywords={...},            % if you want to delete keywords from the given language
  escapeinside={\%*}{*)},          % if you want to add LaTeX within your code
  extendedchars=true,              % lets you use non-ASCII characters; for 8-bits encodings only, does not work with UTF-8
  frame=single,	                   % adds a frame around the code
  keepspaces=true,                 % keeps spaces in text, useful for keeping indentation of code (possibly needs columns=flexible)
  keywordstyle=\color{blue},       % keyword style
  morekeywords={*,...},            % if you want to add more keywords to the set
  numbers=left,                    % where to put the line-numbers; possible values are (none, left, right)
  numbersep=5pt,                   % how far the line-numbers are from the code
  numberstyle=\tiny\color{gray},   % the style that is used for the line-numbers
  rulecolor=\color{black},         % if not set, the frame-color may be changed on line-breaks within not-black text (e.g. comments (green here))
  showspaces=false,                % show spaces everywhere adding particular underscores; it overrides 'showstringspaces'
  showstringspaces=false,          % underline spaces within strings only
  showtabs=false,                  % show tabs within strings adding particular underscores
  stepnumber=2,                    % the step between two line-numbers. If it's 1, each line will be numbered
  stringstyle=\color{cyan},        % string literal style
  tabsize=2,	                   % sets default tabsize to 2 spaces
  title=\lstname,                  % show the filename of files included with \lstinputlisting; also try caption instead of title
  % needed if you want to use UTF-8 Polish chars
  literate={ą}{{\k{a}}}1
  {Ą}{{\k{A}}}1
  {ę}{{\k{e}}}1
  {Ę}{{\k{E}}}1
  {ó}{{\'o}}1
  {Ó}{{\'O}}1
  {ś}{{\'s}}1
  {Ś}{{\'S}}1
  {ł}{{\l{}}}1
  {Ł}{{\L{}}}1
  {ż}{{\.z}}1
  {Ż}{{\.Z}}1
  {ź}{{\'z}}1
  {Ź}{{\'Z}}1
  {ć}{{\'c}}1
  {Ć}{{\'C}}1
  {ń}{{\'n}}1
  {Ń}{{\'N}}1
}
