\documentclass[12pt]{article}
% \usepackage[polish]{babel}
\usepackage{polski} \usepackage{natbib} \usepackage{url}
\usepackage[utf8]{inputenc} \usepackage{amsmath} \usepackage{graphicx}
\usepackage{parskip} \usepackage{fancyhdr} \usepackage{vmargin} \setmarginsrb{3
  cm}{2.5 cm}{3 cm}{2.5 cm}{1 cm}{1.5 cm}{1 cm}{1.5 cm}

\title{Binutils, biblioteki statyczne i dynamiczne} % Title
\author{} % Author
\date{\today} % Date

\makeatletter \let\thetitle\@title \let\theauthor\@author \let\thedate\@date
\makeatother

\pagestyle{fancy} \fancyhf{} \rhead{\theauthor} \lhead{\thetitle}
\cfoot{\thepage}

\begin{document}

%%%%%%%%%%%%%%%%%%%%%%%%%%%%%%%%%%%%%%%%%%%%%%%%%%%%%%%%%%%%%%%%%%%%%%%%%%%%%%%%%%%%%%%%%

\begin{titlepage}
  \centering \vspace*{0.5 cm} \includegraphics[scale = 0.75]{agh.jpg}\\[1.0
  cm] % University Logo
  % \textsc{\LARGE Akademia Górniczo-Hutnicza im. Stanisława Staszica w
  % Krakowie}\\[2.0 cm] % University Name
  \textsc{\Large Programowanie Niskopoziomowe}\\[0.5 cm] % Course Code
  \textsc{\large Konspekt Laboratoryjny}\\[0.5 cm] % Course Name
  \rule{\linewidth}{0.2 mm} \\[0.4 cm]
  { \huge \bfseries \thetitle}\\
  \rule{\linewidth}{0.2 mm} \\[1.5 cm]

  \begin{minipage}{0.4\textwidth}
    \begin{flushleft} \large \vspace{3cm}
      \emph{Autorzy:}\\
      Gabriel Górski\newline Robert Gałat % Your Student Number
    \end{flushleft}
  \end{minipage}~
  \begin{minipage}{0.4\textwidth}
    \begin{flushright} \large
    \end{flushright}
  \end{minipage}\\[2 cm]

  {\large \thedate}\\[2 cm]

  \vfill

\end{titlepage}

%%%%%%%%%%%%%%%%%%%%%%%%%%%%%%%%%%%%%%%%%%%%%%%%%%%%%%%%%%%%%%%%%%%%%%%%%%%%%%%%%%%%%%%%%

\tableofcontents
\pagebreak

%%%%%%%%%%%%%%%%%%%%%%%%%%%%%%%%%%%%%%%%%%%%%%%%%%%%%%%%%%%%%%%%%%%%%%%%%%%%%%%%%%%%%%%%%


\section{Informacje do zadań}

\section{Zadania}
\subsection{Biblioteki statyczne}
\begin{itemize}
\item Celem zadania jest uzupełnienie pliku \textbf{run.sh} tak, aby
  umożliwał on kompilację biblioteki statycznej, oraz zlinkowanie projektu do
  programu wynikowego. [\textbf{1a}]
\item Celem zadania jest uzupełnienie pliku \textbf{run.sh} tak, aby
  stworzyć biblioteki statyczne oraz zlinkować je z \textit{głównym} plikiem
  obiektowym. Czy zauważasz coś ciekawego? Jeśli tak, to czy potrafisz to
  wyjaśnić? [\textbf{1b}]
\end{itemize}
\subsection{Biblioteki dynamiczne}
\begin{itemize}
\item W tym zadaniu należy utworzyć bibliotekę dynamiczną, zlinkować wobec niej
  plik obiektowy, a następnie otrzymany plik wykonywalny należy uruchomić ---
  pamiętaj o odpowiednich flagach kompilacji i linkowania! [\textbf{2}]
\item Celem zadania jest podmienienie implementacji funkcji która była w
  bibliotece z poprzedniego zadania.

  Należy to zrobić bez bez modyfikacji pliku wykonywalnego z poprzedniego
  zadania tj. poprzez wykorzystanie funkcjonalności linkera dynamicznego.

  Wprowadź własną implementację tej funkcji. [\textbf{3}]
\end{itemize}
\subsection{Binutils}
\begin{itemize}
\item W tym zadaniu należy dokonać kompilacji pliku \textit{relocatableFile.c}
  a następnie przeanalizować wygenerowany plik binarny programem \textit{nm}
  oraz \textit{objdump} [\textbf{3}]
\end{itemize}

\subsection{Pluginy i dynamiczne ładowanie}
\begin{itemize}
\item Celem zadania jest uzupełnienie pliku \textit{main.c} w taki sposób aby
  uruchomić funkcję z biblioteki libgoo.so, która powinna zostać załadowana w
  czasie działania programu. [\textbf{5}]

\item Celem zadania jest uzupełnienie brakujących części obsługi pluginów,
  oraz napisanie własnego pluginu, wzorując się na przygotowanym przykładzie

  Do uzupełnienia są następujące funkcje:[\textbf{6}]

  \begin{itemize}
  \item apply\_hook() \{PluginManager/PluginManager.c\}
  \item initPlugi() \{PluginManager/PluginLoader.c\}
  \end{itemize}
\end{itemize}

\end{document}
