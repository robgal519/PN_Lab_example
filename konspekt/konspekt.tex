\documentclass[12pt]{article}
%\usepackage[polish]{babel}
\usepackage{polski}
\usepackage{natbib}
\usepackage{url}
\usepackage[utf8]{inputenc}
\usepackage{amsmath}
\usepackage{graphicx}
\usepackage{parskip}
\usepackage{fancyhdr}
\usepackage{vmargin}
\setmarginsrb{3 cm}{2.5 cm}{3 cm}{2.5 cm}{1 cm}{1.5 cm}{1 cm}{1.5 cm}

\title{Binutils, biblioteki statyczne i dynamiczne}								% Title
\author{}								% Author
\date{\today}											% Date

\makeatletter
\let\thetitle\@title
\let\theauthor\@author
\let\thedate\@date
\makeatother

\pagestyle{fancy}
\fancyhf{}
\rhead{\theauthor}
\lhead{\thetitle}
\cfoot{\thepage}

\begin{document}

%%%%%%%%%%%%%%%%%%%%%%%%%%%%%%%%%%%%%%%%%%%%%%%%%%%%%%%%%%%%%%%%%%%%%%%%%%%%%%%%%%%%%%%%%

\begin{titlepage}
	\centering
    \vspace*{0.5 cm}
    \includegraphics[scale = 0.75]{agh.jpg}\\[1.0 cm]	% University Logo
   % \textsc{\LARGE Akademia Górniczo-Hutnicza im. Stanisława Staszica w Krakowie}\\[2.0 cm]	% University Name
	\textsc{\Large Programowanie Niskopoziomowe}\\[0.5 cm]				% Course Code
	\textsc{\large Konspekt Laboratoryjny}\\[0.5 cm]				% Course Name
	\rule{\linewidth}{0.2 mm} \\[0.4 cm]
	{ \huge \bfseries \thetitle}\\
	\rule{\linewidth}{0.2 mm} \\[1.5 cm]
	
	\begin{minipage}{0.4\textwidth}
		\begin{flushleft} \large
      \vspace{3cm}
			\emph{Autorzy:}\\
			Gabriel Górski\newline
			Robert Gałat						% Your Student Number
			\end{flushleft}
			\end{minipage}~
			\begin{minipage}{0.4\textwidth}
			\begin{flushright} \large
		\end{flushright}
	\end{minipage}\\[2 cm]
	
	{\large \thedate}\\[2 cm]
 
	\vfill
	
\end{titlepage}

%%%%%%%%%%%%%%%%%%%%%%%%%%%%%%%%%%%%%%%%%%%%%%%%%%%%%%%%%%%%%%%%%%%%%%%%%%%%%%%%%%%%%%%%%

\tableofcontents
\pagebreak

%%%%%%%%%%%%%%%%%%%%%%%%%%%%%%%%%%%%%%%%%%%%%%%%%%%%%%%%%%%%%%%%%%%%%%%%%%%%%%%%%%%%%%%%%


\section{Informacje do zadań}

\section{Zadania}
\subsection{Binutils}
\subsubsection{nm}
Korzystajac z programu 'nm' znajdź jakie symbole znajdują się w pliku
lib.......TODO........so.
\subsubsection{objdump}
Z pomocą programu objdump dokonaj deasemblacji biblioteki
lib..........TODO.......so następnie użyj zdobytej wiedzy, aby program main
wypisał ``hello'' na standardowe wyjście
\subsection{Biblioteki statyczne}
Celem zadania jest uzupełnienie pliku 'makefile' tak, aby umożliwał on
kompilacje biblioteki statycznej, oraz zlinkowanie projektu do programu
wynikowego.
\subsection{Zmienne środowiskowe i biblioteki dynamiczne}
\subsubsection{LD\_LIBRARY\_PATH}\label{ex1}
Celem zadania jes napisanie makefile tak aby program main..............
skompilował się i uruchomił bez błędów.
\subsubsection{LD\_DEBUG}
Korzystając z zadania \ref{ex1} oraz programu grep znajdź linijkę w której
ładowana jest biblioteka którą napisano w tym zadaniu.
\subsubsection{LD\_PRELOAD}
Celem zadania jest napisanie biblioteki dynamicznej implementującej funkcje
biblioteki .......TODO..... 'hello(char*)' - jej zachowanie ma powodować że każdy
wyświetlony tekst będzie w kolorze zielonym (.......TODO......) Należy równierz
pamiętać aby przywrócić domyślne ustawienia koloru po wypisaniu tekstu na
standardowe wyjście.

\subsection{Plugin}
\subsubsection{Manager pluginów}
Uzupełnij program main, tak aby poprawnie odczytywał plugin aaaaa.so
\subsubsection{Plugin}
Korzystając z poprzedniego zadania napisz plugin który wypisze na standardowe
wyjście wiadomość powitalną

\end{document}