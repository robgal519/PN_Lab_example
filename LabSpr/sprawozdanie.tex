\documentclass[12pt]{article}
\usepackage{polski} \usepackage{natbib} \usepackage{url}
\usepackage[utf8]{inputenc} \usepackage{amsmath} \usepackage{graphicx}
\usepackage{parskip} \usepackage{fancyhdr} \usepackage{vmargin} \setmarginsrb{3
  cm}{2.5 cm}{3 cm}{2.5 cm}{1 cm}{1.5 cm}{1 cm}{1.5 cm} \usepackage{listings}

\lstset{basicstyle=\ttfamily, showstringspaces=false, commentstyle=\color{red},
  keywordstyle=\color{blue} }

\title{Binutils, biblioteki statyczne i dynamiczne} % Title
\author{} % Author
\date{15 V 2018} % Date

\makeatletter \let\thetitle\@title \let\theauthor\@author \let\thedate\@date
\makeatother

\pagestyle{fancy} \fancyhf{} \rhead{\theauthor} \lhead{\thetitle}
\cfoot{\thepage}

\begin{document}

%%%%%%%%%%%%%%%%%%%%%%%%%%%%%%%%%%%%%%%%%%%%%%%%%%%%%%%%%%%%%%%%%%%%%%%%%%%%%%%%%%%%%%%%%

\begin{titlepage}
  \centering \vspace*{0.5 cm} \includegraphics[scale = 0.75]{agh.jpg}\\[1.0
  cm] % University Logo
  % \textsc{\LARGE Akademia Górniczo-Hutnicza im. Stanisława Staszica w
  % Krakowie}\\[2.0 cm] % University Name
  \textsc{\Large Programowanie Niskopoziomowe}\\[0.5 cm] % Course Code
  \textsc{\large Sprawozdanie\\Laboratoryjne}\\[0.5 cm] % Course Name
  \rule{\linewidth}{0.2 mm} \\[0.4 cm]
  { \huge \bfseries \thetitle}\\
  \rule{\linewidth}{0.2 mm} \\[1.5 cm]

  \begin{minipage}{0.4\textwidth}
    \begin{flushleft} \large \vspace{3cm}
      \emph{Autorzy:}\\
      Gabriel Górski\newline Robert Gałat % Your Student Number
    \end{flushleft}
  \end{minipage}~
  \begin{minipage}{0.4\textwidth}
    \begin{flushright} \large
    \end{flushright}
  \end{minipage}\\[1 cm]
  {\large \thedate}
  \pagebreak

\end{titlepage}

%%%%%%%%%%%%%%%%%%%%%%%%%%%%%%%%%%%%%%%%%%%%%%%%%%%%%%%%%%%%%%%%%%%%%%%%%%%%%%%%%%%%%%%%%

%\tableofcontents
%\pagebreak

%%%%%%%%%%%%%%%%%%%%%%%%%%%%%%%%%%%%%%%%%%%%%%%%%%%%%%%%%%%%%%%%%%%%%%%%%%%%%%%%%%%%%%%%%

\section{Wstęp}
Przy przygotowaniu laboratoriów staraliśmy się tworzyć zadania intensywnie
wykorzystujące zagadnienia przekazane studentom w czasie naszych seminariów.
Uważamy, że zadanie się udało, chociaż nie udało się uniknąć pewnych
niedociągnięć.

\section{Zadania}
\subsection{Biblioteki statyczne}
\begin{itemize}
\item Celem zadania jest uzupełnienie pliku \textbf{run.sh} tak, aby umożliwał
  on kompilację biblioteki statycznej, oraz zlinkowanie projektu do programu
  wynikowego. [\textbf{1a}]\\ \\
  \textbf{Zalety i wady}:
  \begin{itemize}
  \item[$+$] Prosto i szybko wprowadza studenta w tematykę tworzenia bibliotek
    statycznych
  \item[$+$] Wprowadzenie do najbardziej użytecznych i fundamentalnych przy
    tworzeniu bibliotek flag \textit{gcc}
  \item[$-$] Zadanie jest \textit{odtwórcze} --- polega jedynie na przepisaniu
    fragmentu konspektu do pliku skryptu
  \end{itemize}
\item Celem zadania jest uzupełnienie pliku \textbf{run.sh} tak, aby stworzyć
  biblioteki statyczne oraz zlinkować je z \textit{głównym} plikiem obiektowym.
  Czy zauważasz coś ciekawego? Jeśli tak, to czy potrafisz to wyjaśnić?
  [\textbf{1b}]\\ \\
  \textbf{Zalety i wady}:
  \begin{itemize}
  \item[$+$] Zadanie jasno pokazuje znaczenie kolejności linkowania bibliotek
    statycznych prowokując studenta do myślenia
  \item[$-$] Częściowy brak zrozumienia celu zadania wywołujący konsternację u
    studentów --- konieczne było dodatkowe tłumaczenie \textit{(Wniosek:
      zwiększenie stopnia interaktywności w zadaniu, przykładowo dodanie
      graficznej interpretacji listy brakujących referencji z kroku na krok)}
  \end{itemize}
\end{itemize}
\subsection{Biblioteki współdzielone}
\begin{itemize}
\item W tym zadaniu należy utworzyć bibliotekę dynamiczną, zlinkować wobec niej
  plik obiektowy, a następnie otrzymany plik wykonywalny należy uruchomić ---
  pamiętaj o odpowiednich flagach kompilacji i linkowania! [\textbf{2}]\\ \\
  \textbf{Zalety i wady}:
  \begin{itemize}
  \item[$+$] Łączy problematykę linkowania dynamicznego (zmienne środowiskowe) z
    tworzeniem i używaniem bibliotek dynamicznych
  \item[$+$] Utrwalenie i konfrontacja informacji zdobytych podczas poprzednich
    zadań - uświadomienie sobie różnic między typami bibliotek i sposobami ich
    tworzenia
  \item[$-$] Nietrafiona konstrukcja konspektu oddzielająca zagadnienia
    bibliotek współdzielonych od zmiennych środowiskowych linkera dynamicznego
    wywoływała konieczność dodatkowego tłumaczenia \textit{(Wniosek: Lepsze
      zgrupowanie informacji zawartych w konspekcie)}
  \end{itemize}
\item Celem zadania jest podmienienie implementacji funkcji która była w
  bibliotece z poprzedniego zadania.

  Należy to zrobić bez modyfikacji pliku wykonywalnego z poprzedniego zadania
  tj. poprzez wykorzystanie funkcjonalności linkera dynamicznego.

  Wprowadź własną implementację tej funkcji. [\textbf{3}]\\ \\
  \textbf{Zalety i wady}:
  \begin{itemize}
  \item[$+$] W przystępny sposób studenci starli się z dalszą obsługą linkera
    dynamicznego i w jaki sposób można \textit{oszukiwać} pliki wykonywalne w
    odniesieniu do dostarczanych im symboli
  \item[$-$] Brak precyzji w wymaganym nazewnictwie plików obiektowych
    powodujący niedziałające współzależności między zadaniami \textit{Wniosek:
      Usunięcie zależności między zadaniami}
  \end{itemize}
\end{itemize}
\subsection{Binutils}
\begin{itemize}
\item W tym zadaniu należy dokonać kompilacji pliku \textit{relocatableFile.c} a
  następnie przeanalizować wygenerowany plik binarny programem \textit{nm} oraz
  \textit{objdump} [\textbf{4}]\\ \\
  \textbf{Zalety i wady}:
  \begin{itemize}
  \item[$+$] Przedstawienie fundamentalnych narzędzi do analizy plików
    binarnych/ obiektowych
  \item[$-$] Spreparowany przykład sugerował kompilację + linkowanie zamiast
    \textit{jedynie} kompilację --- konsternacja studentów
  \item[$-$] Poświęcenie tematowi mniej czasu niż pozostałym zagadnieniom
  \end{itemize}

\end{itemize}

\subsection{Pluginy i dynamiczne ładowanie}
\begin{itemize}
\item Celem zadania jest uzupełnienie pliku \textit{main.c} w taki sposób aby
  uruchomić funkcję z biblioteki libfoo.so, która powinna zostać załadowana w
  czasie działania programu. [\textbf{5}]\\ \\
  \textbf{Zalety i wady}:
  \begin{itemize}
  \item[$+$] Przykład w doskonały i przystępny sposób prezentuje zagadnienia
    związane z dynamicznym ładowaniem bibliotek/symboli
  \end{itemize}

\item Celem zadania jest uzupełnienie brakujących części obsługi pluginów, oraz
  napisanie własnego pluginu, wzorując się na przygotowanym przykładzie

  Do uzupełnienia są następujące funkcje:[\textbf{6}]

  \begin{itemize}
  \item apply\_hook() \{PluginManager/PluginManager.c\}
  \item initPlugin() \{PluginManager/PluginLoader.c\}
  \end{itemize}
  \textbf{Zalety i wady}:
  \begin{itemize}
  \item[$+$] Angażujące zadanie dostarczające informacji o wzorcach
    wykorzystywanych przy budowaniu programów opartych na pluginach
  \item[$-$] Zadanie nieco przytłaczające dla studentów, wymagające dodatkowej
    pomocy ze strony prowadzących
  \end{itemize}

\end{itemize}

\section{Ostateczne uwagi}
Niestety przygotowana ilość materiału okazała się być niewystarczająca na
potrzeby laboratorium. Dodatkowym problemem była znaczna różnica poziomów
reprezentowana przez niektórych studentów. Kilka wybitnych osób skończyło 1h+
przed czasem, część --- gdyby nie znaczna pomoc ze strony prowadzących --- nie
ukończyłaby wszystkich zadań w czasie zajęć.

Znacznym pozytywem jest jednak to, że wszystkim udało się ukończyć wszystkie
zadania, mając w ten sposób możliwość przejścia przez wszystkie zawiłości i
trudności opracowywanego przez nas tematu.

Wg nas wszystkie najważniejsze zagadnienia dotyczące plików obiektowych (przez
najważniejsze rozumiemy wszystkie te fundamentalne, związane z przyszłą pracą
inżyniera) zostały w projekcie poruszone i liczymy, że także zrozumiane przez
uczestników laboratoriów oraz seminariów.

\end{document}
